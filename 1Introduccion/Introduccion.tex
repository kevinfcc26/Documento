\setcounter{page}{1}


\newpage{\clearpage}
\chapter{ Introducción}

El presente proyecto se refiere al diseño e implementación de un medidor en energía trifásica para sistemas eléctricos no lineales, debido a que los medidores eléctricos actuales están diseñados en su hardware y software para leer el consumo eléctrico, con el problema que estos medidores no tienen en cuenta las perturbaciones que tiene la carga en voltaje y corriente, ya que la mayoría de dispositivos eléctricos que tenemos, no son lineales y la onda fundamental del consumo de estos, no es ideal, es decir que existen perturbaciones en la onda, pero los medidores no las tienen en cuenta. \\

La característica principal de este tipo de medidores, es que al no realizar la lectura correcta de la potencia consumida, las empresas de energía no están realizando un tarificación correcta del KW/h por mes a los hogares, residencias, industrias, etc… Por lo tanto se queda en incertidumbre si la factura que se paga mensual, es mayor o menor de lo que se debería cancelar.\\

Para analizar la problemática es necesario mencionar sus causas. Una de ellas es la evolución de la tecnología. Últimamente la tecnología ha avanzado en gran magnitud y esto permite que muchos errores humanos son fácilmente corregidos por la tecnología, debido a este avance, ya se es posible diseñar un medidor que lea las distorsiones armónicas presentes en la onda, con el fin de que entregue el valor exacto de su consumo.\\

La investigación a esta problemática en el ámbito social y económico, nos permite dar una posible solución al decrecimiento en la economía que ha tenido Colombia en los últimos años, ya que si se logra realizar un medidor para cargas no lineales y si se demuestra que el consumo en dinero es menor, sería un ahorro que los colombianos podrian tener y solventar en una escala mediana la crisis económica que se está viviendo Colombia en distintos sectores de la economía.\\

Profundizar desde el interés académico en hacer el uso de nuevas tecnologías y con ayuda de investigación, lograr descomponer la onda por los coeficientes Fourier y así poder analizar la onda con sus distorsiones. Así mismo, nos interesamos por aportar nuevos descubrimientos o conclusiones a la academia que puedan servir de ayuda para otros proyectos.\\

En el marco de la instrumentación industrial, este proyecto se realiza seleccionando el tipo de tarjeta de desarrollo a utilizar para la medida de señales trifásicas, después de tener los instrumentos necesarios, implementar el sistema de medición, en donde a partir de sus coeficientes de Fourier de voltaje y corriente, se aplica el std IEEE 1459 del 2010, que se basa en la medición de energía en sistemas monofásicos y trifásicos con cargas desbalanceadas con sus respectivas fórmulas. Una vez obtenidos los resultados, se realizará un cuadro de comparación con otros medidores para determinar un porcentaje de error y garantizar los resultados obtenidos. Finalmente corroborados los datos, por medio de una página web se visualiza el valor exacto de consumo mensual a las empresas de energía. \\



\newpage{\clearpage}
\chapter{ Justificación}

La economía en Colombia es un factor que mueve a todo el país y se busca el ahorro o el mejoramiento de este. Por medio del medidor de energía que se plantea en este proyecto, se busca obtener el valor total de la energía consumida por la población y dar un valor exacto en la factura de consumo, esto debido a que el consumo que se registra, se realiza con medidores lineales a sistemas de cargas no lineales, por lo tanto, las pérdidas a la red no se contemplan y hace que en ocasiones el consumidor tenga que pagar por energía que no ha consumido o no pague toda la energía que consumió.\\

Con la norma IEEE STD 1459 del 2010, se obtienen valores más cercanos al consumo exacto y al suministro de potencia de los hogares y en la industria, con esto es posible implementar estudios para mejorar el consumo de energía de todos los componentes de potencia creando un impacto ambiental mejor, aprovechando en su totalidad la generación de energía de todas las centrales importantes en el país.\\
 
Al tener valores más exactos sobre el consumo de energía eléctrica en los hogares y en la industria, se puede generar una cultura social con buenos hábitos de consumo eléctrico, creando conciencia de la cantidad de energía que se está desperdiciando en labores cotidianas.\\

Actualmente en los sectores comerciales no existen medidores que consideren los armónicos presentes en la señales de energía, por esta razón, es viable implementar un medidor que los considere en su medición, ya que éste tendría una alta demanda en el mercado. \\

Para que el medidor trate y analice señales con ondas no sinusoidales y cargas desbalanceadas, se requiere fortalecer los conceptos de potencias energéticas y adquirir nuevos conocimientos acerca de educación energética, para que después se puedan implementar herramientas como smart grids o redes inteligentes donde la energía se pueda generar de otras fuentes y la medición siga siendo precisa y exacta.


\newpage{\clearpage}
\chapter{ Objetivos}
\section{Objetivo General}

Implementar un medidor de energía trifásica para sistemas eléctricos no lineales basados en la norma IEEE STD 1459 del 2010 para el manejo de cargas desbalanceadas y ondas no sinusoidales.

\section{Objetivos Específicos}
\begin{itemize} 
\item[•]Determinar los parámetros matemáticos y electrónicos a considerar para el diseño propuesto de acuerdo con el estándar IEEE 1459-2010. 

\item[•]	Diseñar el sistema electrónico y digital para el sistema de medida propuesto, junto con un software que realice el tratamiento digital de las señales requeridas.

\item[•] Implementar el medidor diseñado con la arquitectura y software desarrollado.

\item[•] Desarrollar un banco y conjunto de pruebas para comparar los resultados del CIRCUTOR y el AEMC 45b y verificar la trazabilidad del sistema implementado. 
 
\end{itemize}



