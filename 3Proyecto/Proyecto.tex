\newpage{\cleardoublepage}
\chapter{Diseño y ejecución del proyecto}


\section{Ejemplo de presentaci\'{o}n y citaci\'{o}n de tablas y cuadros}

De esta participaci\'{o}n aproximadamente el 60 \% proviene de biomasa
(Tabla \ref{EMundo1}).

\begin{center}
\begin{threeparttable}
\centering%
\caption{Participaci\'{o}n de las energ\'{\i}as renovables en el suministro
total de energ\'{\i}a primaria .}\label{EMundo1}
\begin{tabular}{|l|c|c|}\hline
&\multicolumn{2}{c|}{Participaci\'{o}n en el suministro de energ\'{\i}a primaria /\% (Mtoe)\;$\tnote{1}$}\\\cline{2-3}%
\arr{Region}&Energ\'{\i}as renovables &Participaci\'{o}n de la biomasa\\\hline%
Latinoam\'{e}rica&28,9 (140)&62,4 (87,4)\\\hline%
\:Colombia&27,7 (7,6)&54,4 (4,1)\\\hline%
Alemania&3,8 (13,2)&65,8 (8,7)\\\hline%
Mundial&13,1 (1404,0)&79,4 (1114,8)\\\hline
\end{tabular}
\begin{tablenotes}
\item[1] \footnotesize{1 kg oe=10000 kcal=41,868 MJ}
\end{tablenotes}
\end{threeparttable}
\end{center}

NOTA: en el caso en que el contenido de la tabla o cuadro sea muy extenso, se puede cambiar el tama\~{n}o de la letra, siempre y cuando \'{e}sta sea visible por el lector.\\

