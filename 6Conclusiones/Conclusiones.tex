{\clearpage}
\chapter{ Conclusiones}

\begin{itemize}
    \itemsep0em
    \item El proyecto propuesto, como ya se explicó, consiste en cuatro fases, mediante cada una de ellas se obtiene información de mucha utilidad para cumplir los objetivos de este trabajo. Como primera fase, se explican los parámetros, variables y formulas necesarias para emplear la norma IEEE14-59-2010 (1459-2010 IEEE Standard Definitions for the Measurement of Electric Power Quantities Under Sinusoidal, Nonsinusoidal, Balanced, or Unbalanced Conditions., n.d.), la cuál es la base para lograr el objetivo principal del proyecto. Define la forma en la que se debe medir cualquier tipo de carga monfásica y trifásica; la siguiente etapa trata de la identifiación de un procesador que lea y descomponga la señal de voltaje y corriente de cada fase y así poder hacer el tratamiento de datos con el estandar IEEE 1459; luego continúa la implementación del hardware garantizando una conexión segura y optima; posteriormente se desarrolla un software acorde al proyecto que permita tomar las mediciones, realizar el tratamiento de datos y enviarlos al servidor para al final ser visualizados por medio de una página web; con esta información, se crea un analísis de datos que resume toda la información obtenida y procesada donde se muestra un escenario de la vida cotidiana y finalmente permite ver las distorsiones que la carga presenta y así obtener una medición más precisa que es lo que se busca en este proyecto.
    \item Por otro lado, se planteo una métodología que describe el paso a paso de las actividades a desarrollar y llevar a cabo la implementación de un medidor de energía trifásica para sistemas eléctricos no lineales basado en la norma IEEE14-59-2010 (1459-2010 IEEE Standard Definitions for the Measurement of Electric Power Quantities Under Sinusoidal, Nonsinusoidal, Balanced, or Unbalanced Conditions., n.d.). Asimismo, se define y se suministra las herramientas necesarias para la recopilación y estructuración del medidor, se precisan los objetivos y los resultados que se conseguirán al final de cada fase.
    \item Tanto el proyecto como la métodología se implementaron con un foco orientado a la optimización y mejoramiento de la medición energética y debido a su conectividad con internet, abre una gran oportunidad para que sistemas de redes eléctricas inteligentes o fuentes de energía renovable puedan utilizar este sistema de medición.
    \item Cómo se evidenció en la primera prueba del medidor, al momento de medir el bombillo con un consumo de 9 vatios, el medidor presentó un alto porcentaje de error en su medición, por lo tanto se optó por realizar pruebas con cargas que tuvieran un mayor consumo y con cargas de 1 A, el medidor empieza a funcionar de manera optima, por lo tanto el rango de medición del medidor es de 1 A rms a 15 A rms.
    \item El medidor se diseñó con la shunt propuesta debido a que fue la mas adsequible en el momento, sin embargo, la ventaja de este medidor es que se puede adaptar a cualquier tamaño de medicion tan solo cambiando la resistencia shunt y asegurando un voltaje de entrada en la dsp de +- 31.25mV haciendo el dispositivo muy versatil para implementar en cualquier sector.
    \item El ade-7978 no realiza el análisis de estandar IEEE14-59-2010 (1459-2010 IEEE Standard Definitions for the Measurement of Electric Power Quantities Under Sinusoidal, Nonsinusoidal, Balanced, or Unbalanced Conditions., n.d.), sin embargo, entrega los datos necesarios para realizar el tratamiento de datos a travez de un procesador de datos y así cumplir con el estandar.
    \item Se optó por usar el lenguaje de progrmación c++ ya que es un lenguaje de programación de bajo nivel, permite acceder mejor a los registros y hacer un mejor uso de la memoria en el procesador. En el desarrollo de la página web, se usó angular ya que este cuenta con el lenguaje Typescript y nos permite tipar los datos con el fin de tener un sistema robusto.
    \item El medidor tiene conectvidad a internet, por lo tanto el proyecto cuenta con IoT (internet de las cosas) y esto implica que el análisis que tuvo en su arquitectura de software fuera complejo. Para un mejor desempeño es necesario contar con  altos conocimientos en arquitectura de software para garantizar que no halla perdida de datos, sin embargo no se posee dicho conocimiento por lo que queda cómo mejora del proyecto ya que este presenta pequeñas fallas en su medición y es debido a que la frecuencia con la que el ADE 7978 adquiere los datos es muy alta y no contamos con un procesador que vaya a la misma velocidad y a la vez cuente con tarjeta wifi para enviar los datos al servidor.
    \item Debido a que se terminó el desarrollo del medidor en una epoca de pandemia mundial por el covid-19, no fue posible realizar una comparación de mediciones con el AEMC 45b y el circutor, sin embargo se creó un banco de datos y pruebas, el cuál tiene todas las mediciones que se tomaron; posteriormente este banco permite a futuro realizar una comparación con los medidores de referencia propuestos.
    \item Durante el desarrollo del proyecto, se evidenció que la planeación que se hizo en el anteproyecto estuvo mal diseñada ya que se tuvieron inconvenientes como el de escoger el dispositivo correcto que cumpliera con la norma IEEE 1459, tiempo de compra y respuesta por parte del FODEIN para la financiación del proyecto, el lenguaje de programación, protocolo de comunicación y representación de los datos del dispositivo. Por lo tanto, los inconvenientes ya mencionados impactaron en el tiempo de entrega del proyecto y en su complejidad. El objetivo principal es emplear la norma IEEE 1459 en un medidor de energía, sin embargo, la arquitectura de hardware y software paso a ser más compleja debido a que los datos se mostraron por medio de una página web en donde se tuvieron que analizar factores como, velocidad de ejecución, tiempo de respuesta en envío y lectura de datos y comunicación de dispositivos entre otros. Acorde a todo lo que no se tuvo en cuenta en la planeación, se llegó a la conclusión que este proyecto de grado cumple con todos los requerimientos que una tesis de pregrado y postgrado a su vez requieren.
\end{itemize}
\chapter{ Trabajo Futuro}
Como continuación de este proyecto de grado y como en cualquier otro pyecto de investigación, se extienden distintas líneas de investigación que quedan abiertas y en las que es posible continuar trabajando. Durante la ejecución de este proyecto, se han encontrado algunas mejoras futuras que se han dejado abiertas; alguna de ellas, estás relacionadas con este trabajo de tesis y son las interrogantes que se han ido generando mediante la ejecución del proyecto. Otras son más globales, sin embargo no son objeto de este proyecto de grado; estas mejoras o líneas de investigación pueden servir para retomarlas posteriormente o como opción para otros investigadores de realizar un proyecto de grado a partir de este. \\

A continuación se listan algunos trabajos futuos que se pueden realizar apartir de este de los resultados de esta investigación o que, por enfocarnos en otros alcances distintos a los objetivos principales, no han podido ser aplicados con la suficiente profundidad. Igualmente, se recomienda algunos desarrollos especificos para apoyar y mejorar el medidor y la métodología propuestos. \\
Entre los posibles trabajos futuros se sobresalen los siguientes:

\begin{itemize}
    \itemsep0em
    \item Realizar una investigación y busqueda de un micro procesador que permita obtener los datos de la dsp a medida que esta va refrescando sus lecturas con el fin de tener la representación de la señal sinusiodal de voltaje y corriente.
    \item Crear una aplicación móvil en dónde los usuarios puedan ver el consumo de energía de una manera más rápida en vez de ingresar a la página web.
    \item Mejorar la arquitectura actual a una donde muchos clientes puedan usar este sistema de medición y este pueda responder de manera optima.
    \item Implementar un respaldo de datos localmente en dado caso que el internet falle y este pueda ser sincronizado a la base de datos en la nube, una vez se restaure el internet.
\end{itemize}