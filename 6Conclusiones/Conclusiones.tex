{\clearpage}
\chapter{ Conclusiones}

\begin{itemize}
    \itemsep0em
    \item El proyecto propuesto, como ya se explicó, consiste en tres fases, mediante cada una de ellas se obtiene información de mucha utilidad para cumplir los objetivos de este trabajo. Como primera fase, se explica los parámetros, variables y formulas necesarias para emplear la norma IEEE14-59-2010 (1459-2010 IEEE Standard Definitions for the Measurement of Electric Power Quantities Under Sinusoidal, Nonsinusoidal, Balanced, or Unbalanced Conditions., n.d.), la cuál es la base para lograr el objetivo principal del proyecto. Define la forma en la que se debe medir cualquier tipo de carga monfásica y trifásica; la siguiente etapa trata de la identifiación de un procesador que lea y descomponga la señal de voltaje y corriente de cada fase y así poder hacer el tratamiento de datos con el estandar IEEE 1459; luego continúa la implementación del hardware garantizando una conexión segura y optima; posteriormente se desarrolla un software acorde al proyecto que permita tomar las mediciones, realizar el tratamiento de datos y enviarlos al servidor para al finar ser visualizados por medio de una página web; con esta información, se crea un analísis de datos que resume toda la información obtenida y procesada donde se muestra un escenario de la vida cotidiana y finalmente permite ver las distorsiones que la carga presenta y así obtener una medición más precisa que es lo que se busca en este proyecto.
    \item Por otro lado, se plantea una metodología que describe el paso a paso de las actividades a desarrollar y llevar a cabo y la implementación de un medidor de energía trifásica para sistemas eléctricos no lineales basado en la norma IEEE14-59-2010 (1459-2010 IEEE Standard Definitions for the Measurement of Electric Power Quantities Under Sinusoidal, Nonsinusoidal, Balanced, or Unbalanced Conditions., n.d.). Asimismo, se define y se suministra las herramientas necesarias para la recopilación y estructuración del medidor, se precisan los objetivos y los resultados que se conseguirán al final de cada fase.
    \item Tanto el proyecto como la metodología se implementaron con un foco orientado a la optimización y mejoramiento de la medición energética y debido a su conectividad con internet, abre una gran oportunidad para que sistemas de red eléctricas inteligentes o fuentes de energía renovable puedan utilizar este sistema de medición.
    \item Cómo se evidenció en la primera prueba del medidor, al momento de medir el bombillo con un consumo de 9 vatios, el medidor presentó un alto porcentaje de error en su medición, por lo tanto se optó por realizar pruebas con cargas que tuvieran un mayor consumo y con cargas de 1 amperio, el medidor empieza a funcionar de manera optima, por lo tanto el rango de medición del medidor es de 1 amperio a 15 amperios.
    \item El ade-7978 no realiza el análisis de estandar IEEE14-59-2010 (1459-2010 IEEE Standard Definitions for the Measurement of Electric Power Quantities Under Sinusoidal, Nonsinusoidal, Balanced, or Unbalanced Conditions., n.d.), sin embargo, entrega los datos necesarios para realizar el tratamiento de datos a travez de un procesador y así cumplir con el estandar.
    \item El medidor tiene conectvidad a internet, por lo tanto el proyecto cuenta con IoT (internet de las cosas) y esto implica que el análisis que tuvo en su arquitectura de software fuera complejo. Para un mejor desempeño es necesario contar con  altos conocimientos en arquitectura de software para garantizar que no halla perdida de datos, sin embargo no se posee dicho conocimiento por lo que queda cómo mejora del proyecto ya que este presenta pequeñas fallas en su medición y es debido a que la frecuencia con la que el ADE 7978 adquiere los datos es muy alta y no contamos con un procesador que vaya a la misma velocidad y a la vez cuente con tarjeta wifi para enviar los datos al servidor.
    \item Debido a que se terminó el desarrollo del medidor en una epoca de pandemia mundial por el covid-19, no fue posible realizar una comparación de mediciones con el AEMC 45b y circutor, sin embargo se utilizó cómo referencia un multimetro con el fin de calibrar el medidor.
\end{itemize}
\chapter{ Trabajo Futuro}
Como continuación de este proyecto de grado y como en cualquier otro pyecto de investigación, se extienden distintas líneas de investigación que quedan abiertas y en las que es posible continuar trabajando. Durante la ejecución de este proyecto, se han encontrado algunas mejoras futuras que se han dejado abiertas; alguna de ellas, estás relacionadas con este trabajo de tesis y son las interrogantes que se han ido generando mediante la ejecución del proyecto. Otras son más globales, sin embargo no son objeto de este proyecto de grado; estas mejoras o líneas de investigación pueden servir para retomarlas posteriormente o como opción para otros investigadores de realizar un proyecto de grado a partir de este. \\

A continuación se listan algunos trabajos futuos que se pueden realizar apartir de este de los resultados de esta investigación o que, por enfocarnos en otros alcances distintos a los objetivos principales, no han podido ser aplicados con la suficiente profundidad. Igualmente, se recomienda algunos desarrollos especificos para apoyar y mejorar el medidor y la metodología propuestos.\\
Entre los posibles trabajos futuros se sobresalen los siguientes:

\begin{itemize}
    \itemsep0em
    \item Realizar ....
\end{itemize}