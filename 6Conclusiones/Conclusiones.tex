{\clearpage}
\chapter{Conclusiones}

\begin{itemize}
    \itemsep0em
    \item Cómo se evidenció en la primera prueba del medidor, al momento de medir el bombillo con un consumo de 9 vatios, el medidor presentó un alto porcentaje de error en su medición, por lo tanto se optó por realizar pruebas con cargas que tuvieran un mayor consumo y con cargas de 1 amperio, el medidor empieza a funcionar de manera optima, por lo tanto el rango de medición del medidor es de 1 amperio a 15 amperios.
    \item El ade-7978 no realiza el análisis de estandar IEEE 1459 del 2010, sin embargo entrega los datos necesarios para realizar el tratamiento de datos a travez de un procesador y así cumplir con el estandar.
    \item El medidor tiene conectvidad a internet, por lo tanto el proyecto cuenta con IoT (internet de las cosas) y esto implica que el análisis que tuvo en su arquitectura de software fuera complejo, para un mejor desempeño es necesario contar con  altos conocimientos en arquitectura de software para garantizar que no halla perdida de datos, sin embargo no se posee dicho conocimiento por lo que queda cómo mejora del proyecto ya que este presenta pequeñas fallas en su medición y es debido a que la frecuencia con la que el ADE 7978 adquiere los datos es muy alta y no contamos con un procesador que vaya a la misma velocidad y a la vez cuente con tarjeta wifi para enviar los datos al servidor.
    \item Debido a que se terminó el desarrollo del medidor en una epoca de pandemia mundial por el covid-19, no fue posible realizar una comparación de mediciones con el AEMC 45b y circutor, sin embargo se utilizó cómo referencia un multimetro con el fin de calibrar el medidor.
\chapter{Trabajo Futuro}
Se presentan como una serie de aspectos que se podr\'{\i}an realizar en un futuro para emprender investigaciones similares o fortalecer la investigaci\'{o}n realizada. Deben contemplar las perspectivas de la investigaci\'{o}n, las cuales son sugerencias, proyecciones o alternativas que se presentan para modificar, cambiar o incidir sobre una situaci\'{o}n espec\'{\i}fica o una problem\'{a}tica encontrada. Pueden presentarse como un texto con caracter\'{\i}sticas argumentativas, resultado de una reflexi\'{o}n acerca de la tesis o trabajo de investigaci\'{o}n.\\