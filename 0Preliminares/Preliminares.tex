%CUBIERTA
\begin{center}
\thispagestyle{empty} \vspace*{0cm} \huge\textbf{
Diseño e Implementaci\'{o}n de un Medidor de Energ\'{i}a Trif\'{a}sica para Sistemas El\'{e}ctricos no Lineales. }\\[6.0cm]
\Large\textbf{German Andres Sanchez Motta\\Kevin Fabian Carrillo Carrillo}\\[6.0cm]
\small Universidad Santo Tomas\\
Facultad de Ingenier\'{i}a Electr\'{o}nica\\
Bogot\'{a}, Colombia\\
2018\\
\end{center}

\newpage{\pagestyle{empty}\clearpage}

%PORTADA
\newpage
\begin{center}
\thispagestyle{empty} \vspace*{0cm} \huge\textbf{
Diseño e Implementaci\'{o}n de un Medidor de Energ\'{i}a Trif\'{a}sica para Sistemas El\'{e}ctricos no Lineales. }\\[2.5cm]
\Large\textbf{German Andres Sanchez Motta\\Kevin Fabian Carrillo Carrillo}\\[2.5cm]
\small Monograf\'{i}a de proyecto de grado para optar al
t\'{\i}tulo de:\\
\textbf{Ingeniero Electr\'{o}nico }\\[2.5cm]

Director:\\
Ing. Edwin Francisco Forero Garcia, M.Sc.\\[3.7cm]




Universidad Santo Tomas\\
Facultad de Ingenier\'{i}a Electr\'{o}nica \\
Bogot\'{a}, Colombia\\
2018\\
\end{center}


%DEDICATORIA
\newpage{\clearpage}
\thispagestyle{empty} \textbf{}\normalsize
\\\\\\%
\textbf{\LARGE Dedicatoria}\\[4.0cm]
\setcounter{page}{3}
\addcontentsline{toc}{chapter}{\numberline{}Dedicatoria}\\\\

\begin{flushright}
\begin{minipage}{8cm}
    \noindent
        \small
		Principalmente a Dios y después a nuestros padres, quienes fueron de gran ayuda y confianza incondicional en esta trayectoria de formación personal y profesional.  \\[4.0cm]\\
        
\end{minipage}
\end{flushright}


%AGRADECIMIENTOS
\newpage{\clearpage}
\thispagestyle{empty} \textbf{}\normalsize
\\\\\\%
\textbf{\LARGE Agradecimientos}
\setcounter{page}{5}
\addcontentsline{toc}{chapter}{\numberline{}Agradecimientos}\\\\\\
Este proyecto ha sido realizado bajo el acompañamiento y dirección de el ingeniero Edwin Francisco Forero Garcia, M.Sc, por habernos brindado su tiempo, conocimientos, paciencia, le expresamos nuestro más sincero agradecimiento por hacer posible la realización de este proyecto.\\\\
A nuestros padres por habernos forjado como las personas que somos en la actualidad; muchos de nuestros logros se los debemos a ustedes en los que se incluye este. Clara Ines Motta Almario, German Sanchez Pinto, Nelcy Faviola Carrillo Carrillo, William Marquez Carrillo Rey.  \\\\
A nuestras parejas, amigos y colegas, por ayudarnos en este proceso, por aportar algún comentario y crítica a nuestro proyecto, por darnos motivación para culminar este proceso y por los momentos compartidos en espacios académicos y sociales. Julieth Garzon, David Ortiz, Sergio Zambrano, Nicolas Ramirez y Mateo Rojas.  \\\\
A nuestros docentes, quienes desde el primer día que empezamos este proceso nos brindaron sus conocimientos y tiempo para ayudarnos a crecer como personas y profesionales.\\\\\\\\
\begin{flushright}
\textbf{\large{ GRACIAS}}
\end{flushright}



%%RESUMEN 
%\newpage{\clearpage}
%\thispagestyle{empty} \textbf{\LARGE Resumen}
%\addcontentsline{toc}{chapter}{\numberline{}Resumen}\\\\
%El resumen es una presentaci\'{o}n abreviada y precisa (la NTC 1486 de 2008 recomienda revisar la norma ISO 214 de 1976). Se debe usar una extensi\'{o}n m\'{a}xima de 12 renglones. Se recomienda que este resumen sea anal\'{\i}tico, es decir, que sea completo, con informaci\'{o}n cuantitativa y cualitativa, generalmente incluyendo los siguientes aspectos: objetivos, dise\~{n}o, lugar y circunstancias, pacientes (u objetivo del estudio), intervenci\'{o}n, mediciones y principales resultados, y conclusiones. Al final del resumen se deben usar palabras claves tomadas del texto (m\'{\i}nimo 3 y m\'{a}ximo 7 palabras), las cuales permiten la recuperaci\'{o}n de la informaci\'{o}n.\\
%
%\textbf{\small Palabras clave: (m\'{a}ximo 10 palabras, preferiblemente seleccionadas de las listas internacionales que permitan el indizado cruzado)}.\\
%
%\textbf{Artes}: AAT: Art y Architecture Thesaurus.\\
%
%
%\textbf{\LARGE Abstract}\\\\
%Es el mismo resumen pero traducido al ingl\'{e}s. Se debe usar una extensi\'{o}n m\'{a}xima de 12 renglones. Al final del Abstract se deben traducir las anteriores palabras claves tomadas del texto (m\'{\i}nimo 3 y m\'{a}ximo 7 palabras), llamadas keywords. Es posible incluir el resumen en otro idioma diferente al espa\~{n}ol o al ingl\'{e}s, si se considera como importante dentro del tema tratado en la investigaci\'{o}n, por ejemplo: un trabajo dedicado a problemas ling\"{u}\'{\i}sticos del mandar\'{\i}n seguramente estar\'{\i}a mejor con un resumen en mandar\'{\i}n.\\[2.0cm]
%\textbf{\small Keywords: palabras clave en ingl\'{e}s(m\'{a}ximo 10 palabras, preferiblemente seleccionadas de las listas internacionales que permitan el indizado cruzado)}\\


